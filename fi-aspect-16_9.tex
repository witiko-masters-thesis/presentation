\documentclass[aspectratio=169,t]{beamer}
\usepackage[czech]{babel}
\usepackage[utf8]{inputenc}
\usepackage[T1]{fontenc}
\usepackage{booktabs}
\usepackage{diagrams}
\usetheme[workplace=fi]{MU}
\title{Vektorové reprezentace ve vyhledávání znalostí}
\subtitle{Vector Space Representations in Information Retrieval}
\author[V.\ Novotný]{Vít Novotný \\ witiko@mail.muni.cz}
\institute[FI MU]{Faculty of Informatics, Masaryk University}
\date{\today}
\subject{Presentation Subject}
\keywords{the, presentation, keywords}
\date{6.\ února 2017}

% Macro definitions.
\let\abbr\relax
\begin{document}

\begin{frame}[plain]
\maketitle
\end{frame}

\begin{frame}{Obsah}
\tableofcontents
\end{frame}

\section{Úvod}
\begin{frame}{Úvod}
\begin{itemize}
  \item<1-7> V~rámci výzkumné skupiny \abbr{MIR} na \abbr{FI} \abbr{MU} jsem se
    ve spolupráci s~RaRe Technologies s.r.o. zúčastnil druhého ročníku projektu
    \alert<1>{\abbr{TA} \abbr{ČR} Omega}.
  \item<2-7> Cíl byl vyvinout \alert<2-3>{segmentující vyhledávač}
    nestrukturovaných textových dokumentů.
  \only<4-7>{%
  \item<4-7> V rámci projektu jsem dostal možnost prezentovat náš výzkum na \abbr{ACL}~2017.
  \item<5-7> V návaznosti na výzkum prezentovaný na \abbr{ACL}~2017 jsem
    provedl \alert<5-7>{dvojici experimentů}, které jsou předmětem této
    diplomové práce:
  \begin{enumerate}
    \item<6-7> U~vyhledávačů, které musí vždy navrátit celé dokumenty a nikoliv
      pouze segmenty, lze \alert<6>{agregací nalezených segmentů} zlepšit
      kvalitu výsledků oproti hledání bez segmentace.
    \item<7-7> Rozšířením standardního vektorového modelu o neortogonalitu mezi
      bázovými vektory lze \alert<7>{modelovat synonymitu slov} a docílit
      dalšího zlepšení kvality výsledků.
  \end{enumerate}}
\end{itemize}
\only<1-3>{%
  \only<3>{\hspace*{-14.5cm}}%
  \uncover<2-3>{\STqueryingdiagram}}
\only<4-7>{\vspace*{2cm}}
\end{frame}

\section[Short Section 1 Name]{Full Section 1 Name}
\subsection[Short Subsection 1 Name]{Full Subsection 1 Name}

\begin{frame}{Frame Title}{Frame Subtitle}
plain text, \structure{page structure}, \alert{emphasis}
\begin{itemize}
  \item a single-line bullet list item
  \item a bullet list item that is quite long (in order to force a line break),
    which also contains \alert{emphasized text}
  \begin{itemize}
    \item a second-level list item
    \begin{itemize}
      \item a third-level list item
    \end{itemize}
    \alert{\item an emphasized second-level list item}
  \end{itemize}
\end{itemize}
\begin{enumerate}
  \item a numbered list item
  \begin{enumerate}
    \item a second-level list item containing a math expression
      \[ E = mc^2 \]
  \end{enumerate}
\end{enumerate}
\end{frame}

\subsection[Short Subsection 2 Name]{Full Subsection 2 Name}

\begin{frame}{Text Blocks}
text above a block
\begin{block}{Block}
  text
\end{block}
\begin{exampleblock}{Example Block}
  text
\end{exampleblock}
\begin{alertblock}{Emphasized Block}
  text
\end{alertblock}
text below a block\footnote{a footnote with an \url{http://address.edu}}
\end{frame}

\begin{frame}{Figures}
\begin{figure}
  \includegraphics[width=.5\textwidth,height=.5\textheight,keepaspectratio]{cow-black.mps}
  \caption{A Holstein Friesian cow}
\end{figure}
\end{frame}

\subsection[Short Subsection 3 Name]{Full Subsection 3 Name}

\begin{frame}{Tables}
\begin{table}
  \begin{tabular}{llc}
    First Name & Surname & Year of Birth \\ \midrule
    Albert & Einstein & 1879 \\
    Marie & Curie & 1867 \\
    Thomas & Edison & 1847 \\
  \end{tabular}
  \caption{The great minds of the 19th century}
\end{table}
\end{frame}

\makeatletter
\begin{frame}{Automatic Optical Scaling}
\begin{center}
\begin{tabular}{ll}
\Huge \f@family & \Huge \structure{\f@size pt} \\
\huge \f@family & \huge \structure{\f@size pt}  \\
\LARGE \f@family & \LARGE \structure{\f@size pt}  \\
\Large \f@family & \Large \structure{\f@size pt}  \\
\large \f@family & \large \structure{\f@size pt}  \\
\normalsize \f@family & \normalsize \structure{\f@size pt}  \\[-0.95pt]
\small \f@family & \small \structure{\f@size pt}  \\[-1.95pt]
\footnotesize \f@family & \footnotesize \structure{\f@size pt} \\[-2.95pt]
\scriptsize \f@family & \scriptsize \structure{\f@size pt}  \\[-4.95pt]
\tiny \f@family & \tiny \structure{\f@size pt}
\end{tabular}
\end{center}
\end{frame}
\makeatother

\begin{frame}[plain]
\vfill
\centerline{Thank you for your attention!}
\vfill\vfill
\end{frame}

\end{document}
